This paper discusses three possible homogeneity functions that can be used to compare how well different items in a subset of an item set fit together. They can be used to assess how well a clustering algorithm performs in terms of the quality of the resulting clusters. We give arguments for using the entropy-based homogeneity measure over the others. A baseline approach using the Power Iteration Clustering (PIC) algorithm and two variations of a custom greedy approach were proposed for the problem of partitioning a data set into homogeneous areas. The approaches were compared both in terms of resulting homogeneity and in terms of run time performance. The $k$-split greedy algorithm resulted in clusters with a higher homogeneity compared to those using the PIC algorithm. The PIC algorithm had a lower run time for a high number of clusters with a low number of records. Yet, we presented empirical and theoretical results that show that the baseline approach does not scale well in correspondence with the volume of data when compared to our custom technique.